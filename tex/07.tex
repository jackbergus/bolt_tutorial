%\hypertarget{___gatsby}{}
%\hypertarget{gatsby-focus-wrapper}{}
%\href{https://mukulrathi.com/}{}
%
%MUKUL RATHI
%
%\href{https://mukulrathi.com/about-me}{}
%
%About Me
%
%\href{https://mukulrathi.com/blog}{}
%
%Blog
%
%\hypertarget{creating-the-bolt-compiler-part-7}{%
%\section{Creating the Bolt Compiler: Part
%7}\label{creating-the-bolt-compiler-part-7}}

\hypertarget{top-of-page}{%
\chapter{A Protobuf tutorial for OCaml and C++}\label{top-of-page}}

October 03, 2020

\includegraphics[width=\linewidth]{07_files/protobuf.png}

%\hypertarget{october-03-2020}{%
%\section{October 03, 2020}\label{october-03-2020}}
%
%\hypertarget{min-read}{%
%\section{8 min read}\label{min-read}}
%
%\includegraphics[width=6.5625in,height=2.45833in]{07_files/protobuf_002.png}
%
%\hypertarget{series-creating-the-bolt-compiler}{%
%\section{Series: Creating the Bolt
%Compiler}\label{series-creating-the-bolt-compiler}}
%
%\begin{itemize}
%\item
%  { Part 1:
%  }\href{https://mukulrathi.com/create-your-own-programming-language/intro-to-compiler/}{How
%  I wrote my own "proper" programming language}
%\item
%  { Part 2:
%  }\href{https://mukulrathi.com/create-your-own-programming-language/compiler-engineering-structure/}{So
%  how do you structure a compiler project?}
%\item
%  { Part 3:
%  }\href{https://mukulrathi.com/create-your-own-programming-language/parsing-ocamllex-menhir/}{Writing
%  a Lexer and Parser using OCamllex and Menhir}
%\item
%  { Part 4:
%  }\href{https://mukulrathi.com/create-your-own-programming-language/intro-to-type-checking/}{An
%  accessible introduction to type theory and implementing a
%  type-checker}
%\item
%  { Part 5:
%  }\href{https://mukulrathi.com/create-your-own-programming-language/data-race-dataflow-analysis/}{A
%  tutorial on liveness and alias dataflow analysis}
%\item
%  { Part 6:
%  }\href{https://mukulrathi.com/create-your-own-programming-language/lower-language-constructs-to-llvm/}{Desugaring
%  - taking our high-level language and simplifying it!}
%\item
%  \textbf{Part 7: A Protobuf tutorial for OCaml and C++}
%\item
%  { Part 8:
%  }\href{https://mukulrathi.com/create-your-own-programming-language/llvm-ir-cpp-api-tutorial/}{A
%  Complete Guide to LLVM for Programming Language Creators}
%\item
%  { Part 9:
%  }\href{https://mukulrathi.com/create-your-own-programming-language/concurrency-runtime-language-tutorial/}{Implementing
%  Concurrency and our Runtime Library}
%\item
%  { Part 10:
%  }\href{https://mukulrathi.com/create-your-own-programming-language/generics-parametric-polymorphism/}{Generics
%  - adding polymorphism to Bolt}
%\item
%  { Part 11:
%  }\href{https://mukulrathi.com/create-your-own-programming-language/inheritance-method-overriding-vtable/}{Adding
%  Inheritance and Method Overriding to Our Language}
%\end{itemize}
%
%\begin{center}\rule{0.5\linewidth}{0.5pt}\end{center}

Now that we've desugared our language into a simple ``Bolt IR'' that is
close to LLVM IR, we want to generate LLVM IR. One problem though, our
Bolt IR is an OCaml value, but the LLVM API we're using is the native
C++ API. We can't import the value directly into the C++ compiler
backend, so we need to serialise it first into a
\textbf{language-independent} data representation.

Hey there! If you came across this tutorial from Google and don't care
about compilers, don't worry! This tutorial doesn't really involve
anything compiler-specific (it's just the illustrative example). If you
care about OCaml then the first half will be up your street, and if
you're here for C++ you can skip the OCaml section.

\hypertarget{protocol-buffers-schema}{%
\section{\texorpdfstring{\protect\hyperlink{protocol-buffers-schema}{}Protocol
Buffers
Schema}{Protocol Buffers Schema}}\label{protocol-buffers-schema}}

Protocol buffers (aka \emph{protobuf} ) encodes your data as a series of
binary messages according to a given \textbf{schema}. This schema
(written in a \texttt{.proto} file) captures the structure of your data.

Each message contains a number of fields.

Fields are of the form \texttt{modifer\ type\ name\ =\ someIndex}

Each of the fields have a modifier: \texttt{required} /
\texttt{optional} / \texttt{repeated}. The \texttt{repeated} modifier
corresponds to a list/vector of that field. e.g.
\texttt{repeated\ PhoneNumber} represents a value of type
\texttt{List\textless{}PhoneNumber\textgreater{}}.

For example, the following schema would define a Person in a contact
book. Each person has to have a name and id, and optionally could have
an email address. They might have multiple phone numbers (hence
\texttt{repeated\ PhoneNumber}), e.g. for their home, their mobile and
work.

%{example.proto}
%
%Copy

\begin{verbatim}
message Person {
  required string name = 1;
  required int32 id = 2;
  optional string email = 3;
  enum PhoneType {
    MOBILE = 0; 
   HOME = 1;
    WORK = 2;
  }
  message PhoneNumber {
    required string number = 1;
    optional PhoneType type = 2 [default = HOME];
  }
  repeated PhoneNumber phones = 4;
}
\end{verbatim}

Note here within the schema for a \texttt{Person} message, we also
\textbf{number} the fields (\texttt{=1}, \texttt{=2}, \texttt{=3},
\texttt{=4}). This allows us to add fields later without altering the
existing ordering of fields (so you can continue to parse these fields
without having to know about new fields).

Every time we want to add a new ``type'' of field, we define a schema
(just like how you might define a class to add a new type of object). We
can even define a schema within another schema, e.g. within the schema
for a \texttt{Person}, we defined the schema for the
\texttt{PhoneNumber} field.

We can also define the schema for an \textbf{enum} type
\texttt{PhoneType}. This enum acted as a ``tag'' for our phone number.

Now, just as you might want a field to be one of many options (specified
by the enum type), you might want the message to contain exactly one of
many fields.

For example in our compiler we might want to encode an identifier as
\emph{one of two options}. Here we don't want to store data in an index
field if we've tagged it as a \texttt{Variable}.

%{
%\href{https://github.com/mukul-rathi/bolt/blob/master/src/frontend/ir_gen/frontend_ir.mli}{frontend\_ir.mli}}
%
%Copy

\begin{lstlisting}[language=caml,caption={frontend\_ir.mli}]
type identifier =
    | Variable of string
    | ObjField of string * index (* name and field index *)
\end{lstlisting}

In our protobuf, we can add fields for each of these options, and use
the keyword \texttt{oneof} to specify that only one of the fields in
that group will be set at once.

Now, whilst this tells Protobuf that one of those values is set, we have
no way of telling which one of them is set. So we can introduce a
\texttt{tag} enum field, and query its value (\texttt{Variable} or
\texttt{ObjField}) when deserialising the object.

%Copy

\begin{verbatim}
message identifier {
  enum _tag {
    Variable_tag = 1;
    ObjField_tag = 2;
  }
  message _ObjField {
    required string objName = 1;
    required int64 fieldIndex = 2;
  }
  required _tag tag = 1;
  oneof value {
    string Variable = 2;
    _ObjField ObjField = 3;
  }
}
\end{verbatim}

Note since the constructor \texttt{Variable\ of\ string} has only one
argument of type \texttt{string}, we can just set the type of that field
to \texttt{string}. We equally could have defined a message
\texttt{\_Variable} with schema:

%Copy

\begin{verbatim}
message _Variable {
    required string varName = 1;
  }
  ...
  oneof value {
    _Variable Variable = 2;
    _ObjField ObjField = 3;
  }
\end{verbatim}

\hypertarget{mapping-ocaml-type-definitions-to-protobuf-schema}{%
\section{\texorpdfstring{\protect\hyperlink{mapping-ocaml-type-definitions-to-protobuf-schema}{}Mapping
OCaml Type definitions to Protobuf
Schema}{Mapping OCaml Type definitions to Protobuf Schema}}\label{mapping-ocaml-type-definitions-to-protobuf-schema}}

Now we've looked at the \texttt{identifier} schema, let's flesh out the
rest of the frontend ``Bolt IR'' schema.

If you notice, whenever we have an OCaml variant type like
\texttt{identifier} we have a straightforward formula to specify the
corresponding schema. We:

\begin{itemize}
\tightlist
\item
  Add a tag field to specify the constructor the value has.
\item
  Define a message schema if a constructor has multiple arguments e.g
  for \texttt{ObjField}. If the constructor has only one argument
  (\texttt{Variable}) we don't need to.
\item
  Specify a \texttt{oneof} block containing the fields for each of the
  constructors.
\end{itemize}

We could manually write out all of the schema, but we'd have to rewrite
these every time our language changed. However we have a hidden weapon
up our sleeve - a library that will do this for us!

There's the
\href{https://developers.google.com/protocol-buffers/docs/overview}{full
Protobuf Schema Guide} if you want to learn more about other message
types.

\hypertarget{ppx-deriving-protobuf}{%
\section{\texorpdfstring{\protect\hyperlink{ppx-deriving-protobuf}{}PPX
Deriving Protobuf}{PPX Deriving Protobuf}}\label{ppx-deriving-protobuf}}

OCaml allows libraries to extend its syntax using a \textbf{ppx} hook.
The ppx libraries can preprocess the files using these hooks to extend
the language functionality. So we can tag our files with other
information, such as which types need to have a protobuf schema
generated.

We can update our Dune build file to preprocess our \texttt{ir\_gen}
library with the ppx-deriving protobuf library:

%{
%\href{https://github.com/mukul-rathi/bolt/blob/master/src/frontend/ir_gen/dune}{dune}}
%
%Copy

\begin{lstlisting}[caption={dune}]
(library
 (name ir_gen)
 (public_name ir_gen)
 (libraries core fmt desugaring ast)
 (flags
  (:standard -w -39))
 (preprocess
  (pps ppx_deriving_protobuf bisect_ppx --conditional))
\end{lstlisting}

Note the flags command suppresses warning 39 - ``unused \texttt{rec}
flag'' - since the code the ppx\_deriving\_protobuf library generates
raises a lot of those warnings. I would highly recommend that you also
suppress these warnings!

An aside, \texttt{bisect\_ppx} is the tool we use to calculate test
coverage. It has a \texttt{-\/-conditional} flag since we don't want to
preprocess the file with coverage info if we're not computing the test
coverage.

Telling PPX deriving protobuf that we want to serialise a type
definition is easy - we just stick a \texttt{{[}@@deriving\ protobuf{]}}
at the end of our type definition. For variant types, we have to specify
a \texttt{{[}@key\ 1{]}}, \texttt{{[}@key\ 2{]}} for each of the
variants.

For example, for our \texttt{identifier} type definition in our
\texttt{.mli} file:

%{
%\href{https://github.com/mukul-rathi/bolt/blob/master/src/frontend/ir_gen/frontend_ir.mli}{frontend\_ir.mli}}
%
%Copy

\begin{lstlisting}[language=caml,caption={frontend\_ir.mli}]
type identifier =
  | Variable of string [@key 1]
  | ObjField of string * int [@key 2]  [@@deriving protobuf]
\end{lstlisting}

We do the same thing in the \texttt{.ml} file, except we also specify
the file to which we want to write the protobuf schema.



\begin{lstlisting}[language=caml,caption={frontend\_ir.mli}]
type identifier =
  | Variable of string [@key 1]
  | ObjField of string * int [@key 2][@@deriving protobuf {protoc= "../../frontend_ir.proto"}]
\end{lstlisting}

This path is relative to the src file \texttt{frontend\_ir.ml}. So since
this \texttt{frontend\_ir.ml} file is in the
\texttt{src/frontend/ir\_gen/} folder of the repo, the protobuf schema
file will be written to \texttt{src/frontend\_ir.proto}. If you don't
specify a file path to the \texttt{protoc} argument, then the
\texttt{.proto} file will be written to the \texttt{\_build} folder.

One extra tip, the ppx deriving protobuf library won't serialise lists
of lists. For example you can't have:

%Copy

\begin{lstlisting}[language=caml]
type something = Foo of foo list list [@key 1] [@@deriving protobuf])
\end{lstlisting}

This is because if we have a message schema for \texttt{foo}, then to
get a field of type \texttt{foo\ list} is straightforward - we use
\texttt{repeated\ foo} in our field schema. But we can't say
\texttt{repeated\ repeated\ foo} to get a list of a list of type foo. So
to get around this, you'd have to define another type here:

%Copy

\begin{lstlisting}[language=caml]
type list_of_foo = foo list [@@deriving protobuf]
(*note no @key since not a variant type *)
type something = Foo of list_of_foo list [@key 1] [@@deriving protobuf]
\end{lstlisting}

Finally, to serialise our Bolt IR using this schema, PPX deriving
protobuf provides us with
\texttt{\textless{}type\_name\textgreater{}\_to\_protobuf} serialisation
functions. We use this to get a binary protobuf message that we write to
an output channel (\texttt{out\_chan}) as a sequence of bytes.

%{
%\href{https://github.com/mukul-rathi/bolt/blob/master/src/frontend/ir_gen/ir_gen_protobuf.ml}{ir\_gen\_protobuf.ml}}
%
%Copy

\begin{lstlisting}[caption={ir\_gen\_protobuf.ml},language=caml]
let ir_gen_protobuf program out_chan =
    let protobuf_message = Protobuf.Encoder.encode_exn program_to_protobuf program in
    output_bytes out_chan protobuf_message
\end{lstlisting}

In our overall Bolt compiler pipeline, I write the output to a
\texttt{.ir} file if provided, or to \texttt{stdout}:

%{
%\href{https://github.com/mukul-rathi/bolt/blob/master/src/frontend/compile_program_ir.ml}{compile\_program\_ir.ml}}
%
%Copy

\begin{lstlisting}[caption={{compile\_program\_ir.ml}},language=caml]
...
 match compile_out_file with
    | Some file_name ->
        Out_channel.with_file file_name ~f:(fun file_oc ->
            ir_gen_protobuf program file_oc)
    | None           -> ir_gen_protobuf program stdout )
\end{lstlisting}

\hypertarget{auto-generated-protobuf-schema}{%
\section{\texorpdfstring{\protect\hyperlink{auto-generated-protobuf-schema}{}Auto-generated
Protobuf
Schema}{Auto-generated Protobuf Schema}}\label{auto-generated-protobuf-schema}}

The \href{https://github.com/ocaml-ppx/ppx_deriving_protobuf}{README} of
the repository for the PPX Deriving Library has a thorough explanation
of the mapping. We've already gone through an example of the
autogenerated schema, so the schema shouldn't be too unfamiliar. There
was a slight modification I made though. For
(\texttt{ObjField\ of\ string\ *\ int}) I said the generated output was:

%Copy

\begin{verbatim}
message _ObjField {
    required string objName = 1;
    required int64 fieldIndex = 2;
  }
\end{verbatim}

This is unfortunately not quite true. I put in the objName and
fieldIndex for clarity, but the library doesn't know what the semantic
meaning of \texttt{string\ *\ int} is, so instead it looks like:

%Copy

\begin{verbatim}
message _ObjField {
    required string _0 = 1;
    required int64 _1 = 2;
  }
\end{verbatim}

That's the downside of the autogenerated schema. You get the generic
field names \texttt{\_0}, \texttt{\_1}, \texttt{\_2} and so on instead
of \texttt{objName} and \texttt{fieldIndex}. And for
\texttt{type\ block\_expr\ =\ expr\ list}, you get the field name
\texttt{\_}:

%Copy

\begin{verbatim}
message block_expr {  repeated expr _ = 1;}
\end{verbatim}

\hypertarget{decoding-protobuf-in-c}{%
\section{\texorpdfstring{\protect\hyperlink{decoding-protobuf-in-c}{}Decoding
Protobuf in
C++}{Decoding Protobuf in C++}}\label{decoding-protobuf-in-c}}

Alright, so far we've looked at Protobuf schema definitions and how PPX
Deriving Protobuf encodes messages and generates the schema. Now we need
to decode it using C++. As with the encoding section, we don't need to
know the details of how Protobuf represents our messages in binary.

\hypertarget{generating-protobuf-deserialisation-files}{%
\section{\texorpdfstring{\protect\hyperlink{generating-protobuf-deserialisation-files}{}Generating
Protobuf Deserialisation
Files}{Generating Protobuf Deserialisation Files}}\label{generating-protobuf-deserialisation-files}}

The \texttt{protoc} compiler automatically generates classes and
function definitions from the \texttt{.proto} file: this is exposed in
the \texttt{.pb.h} header file. For \texttt{frontend\_ir.proto}, the
corresponding header file is \texttt{frontend\_ir.pb.h}.

For Bolt, we're building the C++ compiler backend with the build system
Bazel. Rather than manually running the \texttt{protoc} compiler to get
our \texttt{.pb.h} header file and then linking in all the other
generated files, we can instead take advantage of Bazel's integration
with \texttt{protoc} and get Bazel to run \texttt{protoc} during the
build process for us and link it in.

The Bazel build system works with many languages e.g. Java, Dart,
Python, not just C++. So we specify two libraries - a
language-independent \texttt{proto\_library}, and then a specific C++
library (\texttt{cc\_proto}) that wraps around that library:

%{
%\href{https://github.com/mukul-rathi/bolt/blob/master/src/BUILD}{BUILD}}
%
%Copy

\begin{verbatim}
load("@rules_cc//cc:defs.bzl", "cc_library")

# Convention:
# A cc_proto_library that wraps a proto_library named foo_proto
# should be called foo_cc_proto.
cc_proto_library(
    name = "frontend_ir_cc_proto",
    deps = [":frontend_ir_proto"],
    visibility = ["//src/llvm-backend/deserialise_ir:__pkg__"],

)

# Conventions:
# 1. One proto_library rule per .proto file.
# 2. A file named foo.proto will be in a rule named foo_proto.
proto_library(
    name = "frontend_ir_proto",
    srcs = ["frontend_ir.proto"],
)
\end{verbatim}

We include the \texttt{cc\_proto\_library(...)} as a build dependency to
whatever files that need to use the \texttt{.pb.h} file, and Bazel will
compile the protobuf file for us. In our case, this is our
\texttt{deserialise\_ir} library.

%{
%\href{https://github.com/mukul-rathi/bolt/blob/master/src/llvm-backend/deserialise_ir/BUILD}{BUILD}}
%
%Copy

\begin{lstlisting}
load("@rules_cc//cc:defs.bzl", "cc_library")

cc_library(
    name = "deserialise_ir",
    srcs =  glob(["*.cc"]),
    hdrs = glob(["*.h"]),
    visibility = ["//src/llvm-backend:__pkg__", "//src/llvm-backend/llvm_ir_codegen:__pkg__", "//tests/llvm-backend/deserialise_ir:__pkg__"],
    deps = ["//src:frontend_ir_cc_proto", "@llvm"]
)
\end{lstlisting}

\hypertarget{deserialising-a-protobuf-serialised-file}{%
\section{\texorpdfstring{\protect\hyperlink{deserialising-a-protobuf-serialised-file}{}Deserialising
a Protobuf serialised
file}{Deserialising a Protobuf serialised file}}\label{deserialising-a-protobuf-serialised-file}}

The \texttt{frontend\_ir.pb.h} file defines a namespace
\texttt{Frontend\_ir}, with each message mapping to a class. So for our
Bolt program, represented by the \texttt{program} message in our
\texttt{frontend\_ir.proto} file, the corresponding Protobuf class is
\texttt{Frontend\_ir::program}.

To deserialise a message of a given type, we create an object of the
corresponding class. We then call the \texttt{ParseFromIstream} method
which deserialises the message from the input stream and stores it in
the object. This method returns a boolean value indicating
success/failure. We've defined our own custom
\texttt{DeserialiseProtobufException} to handle failure:

%{
%\href{https://github.com/mukul-rathi/bolt/blob/master/src/llvm-backend/deserialise_ir/deserialise_protobuf.cc}{deserialise\_protobuf.cc}}
%
%Copy

\begin{lstlisting}[language=C++,caption={deserialise\_protobuf.cc}]
Frontend_ir::program deserialiseProtobufFile(std::string &filePath) {
  Frontend_ir::program program;
  std::fstream fileIn(filePath, std::ios::in | std::ios::binary);
  ...
  if (!program.ParseFromIstream(&fileIn)) {
    throw DeserialiseProtobufException("Protobuf not deserialised from file.");
  }
  return program;
}
\end{lstlisting}

So we're done!

Well, one caveat\ldots{} this autogenerated class is a \textbf{dumb data
placeholder}. We need to read the data out from the \texttt{program}
object.

\hypertarget{reading-out-protobuf-data-from-a-protobuf-class}{%
\section{\texorpdfstring{\protect\hyperlink{reading-out-protobuf-data-from-a-protobuf-class}{}Reading
out Protobuf data from a protobuf
class}{Reading out Protobuf data from a protobuf class}}\label{reading-out-protobuf-data-from-a-protobuf-class}}

If you try to read the protoc-generated file \texttt{frontend\_ir.pb.h},
you'll realise it's a garguantuan monstrosity, which is not nearly as
nice as the
\href{https://developers.google.com/protocol-buffers/docs/cpptutorial}{standard
example in the official tutorial}. So instead of trying to read the
methods from the file, here is an explanation of the structure of the
header file.

TIP: Make sure you have code-completion set-up in your IDE - it means
you won't need to manually search through \texttt{frontend\_ir.pb.h} for
the right methods.

We'll be deserialising messages using the schema defined below:

%{
%\href{https://github.com/mukul-rathi/bolt/blob/master/src/frontend_ir.proto}{frontend\_ir.proto}}
%
%Copy

\begin{lstlisting}[caption={frontend\_ir.proto}]
package Frontend_ir;

message expr {
  enum _tag {
    Integer_tag = 1;
    Boolean_tag = 2;
    Identifier_tag = 3;
    ...
    IfElse_tag = 11;
    WhileLoop_tag = 12;
    Block_tag = 15;
  }
  message _Identifier {
    required identifier _0 = 1;
    optional lock_type _1 = 2;
  }
  ...
  message _IfElse {
    required expr _0 = 1;
    required block_expr _1 = 2;
    required block_expr _2 = 3;
  }
  message _WhileLoop {
    required expr _0 = 1;
    required block_expr _1 = 2;
  }
  ...
  required _tag tag = 1;
  oneof value {
    int64 Integer = 2;
    bool Boolean = 3;
    _Identifier Identifier = 4;
    ...
    _IfElse IfElse = 12;
    _WhileLoop WhileLoop = 13;
    block_expr Block = 16;
    ...
  }
}

message block_expr {
  repeated expr _ = 1;
}
\end{lstlisting}

\hypertarget{protobuf-message-classes-and-enums}{%
\paragraph{\texorpdfstring{\protect\hyperlink{protobuf-message-classes-and-enums}{}Protobuf
message classes and
enums}{Protobuf message classes and enums}}\label{protobuf-message-classes-and-enums}}

Each Protobuf message definition maps to a class definition. Protobuf
enums map to C++ enums. To give each class/enum a globally unique name,
they are prepended by the package name (\texttt{Frontend\_ir}) and any
classes they're nested in.

So message \texttt{expr} corresponds to class
\texttt{Frontend\_ir\_expr}, and enum \texttt{\_tag} which is nested
within message \texttt{expr} maps to \texttt{Frontend\_ir\_expr\_\_tag}.

In the repo, the \texttt{bin\_op} message also has a nested
\texttt{\_tag} enum, and this maps to
\texttt{Frontend\_ir\_bin\_op\_\_tag} (note how this namespacing means
it doesn't clash with the \texttt{\_tag} definition in the \texttt{expr}
message).

This holds for arbitrary levels of nesting, e.g. the nested message
\texttt{\_IfElse} in the \texttt{expr} message maps to the class
\texttt{Frontend\_ir\_expr\_\_IfElse}.

Specific enum values for the enum \texttt{\_tag} e.g.
\texttt{Integer\_tag} can be referred to as
\texttt{Frontend\_ir\_expr\_\_tag\_Integer\_tag}.

Rather than writing classes\textbackslash enums by concatenating the
package and class names with \texttt{\_}, Protobuf also has a nested
class type alias, so you can write these nested message classes as
\texttt{Frontend\_ir::expr} and \texttt{Frontend\_ir::expr::\_IfElse}.

\hypertarget{accessing-specific-protobuf-fields}{%
\paragraph{\texorpdfstring{\protect\hyperlink{accessing-specific-protobuf-fields}{}Accessing
Specific Protobuf
Fields}{Accessing Specific Protobuf Fields}}\label{accessing-specific-protobuf-fields}}

Each protobuf \emph{required} field \texttt{field\_name} has a
corresponding accessor method \texttt{field\_name()} (where the field
name is converted to lower-case). So the field \texttt{tag} in the
\texttt{expr} message would map to the method \texttt{tag()} in the
\texttt{Frontend\_ir::expr} class, and the field \texttt{Integer} would
map to the method \texttt{integer()}, \texttt{IfElse} to
\texttt{ifelse()} etc.

NB: to reiterate, don't get confused between the field name e.g
\texttt{IfElse} and the type of the field \texttt{\_IfElse} in the
Protobuf message (note the prepended underscore). This is only because
the OCaml PPX deriving protobuf library gave them those names - we could
have chosen less confusing names if we were writing this proto schema
manually.

For \emph{optional} fields, we can access the fields in the same way,
but we also have a \texttt{has\_field\_name()} boolean function to check
if a field is there.

For \emph{repeated} fields, we instead have a
\texttt{field\_name\_size()} function to query the number of items, and
we can access item i using the \texttt{field\_name(i)} method. So for a
field name \texttt{\_1} the corresponding methods are
\texttt{\_1\_size()} and \texttt{\_1(i)}. For field name \texttt{\_} in
the autogenerated schema, the methods would correspondingly be
\texttt{\_\_size()} and \texttt{\_(i)}.

\hypertarget{sanitising-our-frontend-ir}{%
\section{\texorpdfstring{\protect\hyperlink{sanitising-our-frontend-ir}{}Sanitising
our Frontend
IR}{Sanitising our Frontend IR}}\label{sanitising-our-frontend-ir}}

Let's put this in practice by sanitising our protobuf classes into C++
classes directly mapping our OCaml type definitions. Each of the OCaml
\texttt{expr} variants maps to a subclass of an abstract \texttt{ExprIR}
class.

%Copy

\begin{lstlisting}[language=C++]
struct ExprIR {
  virtual ~ExprIR() = default;
  ...
};

struct ExprIfElseIR : public ExprIR {
  std::unique_ptr<ExprIR> condExpr;
  std::vector<std::unique_ptr<ExprIR>> thenBlock;
  std::vector<std::unique_ptr<ExprIR>> elseBlock;
  ExprIfElseIR(const Frontend_ir::expr::_IfElse &expr);
  ...
};
struct ExprWhileLoopIR : public ExprIR {
  std::unique_ptr<ExprIR> condExpr;
  std::vector<std::unique_ptr<ExprIR>> loopExpr;
  ExprWhileLoopIR(const Frontend_ir::expr::_WhileLoop &expr);
  ...
};
...
\end{lstlisting}

I use \texttt{std::unique\_ptr} all over the compiler backend to avoid
explicitly managing memory. You can use standard pointers too - this is
just a personal preference!

Remember how we had a specific \texttt{tag} field to distinguish between
variants of the \texttt{expr} type. We have a \texttt{switch} statement
on the value of the \texttt{tag} field. This \texttt{tag} tells us which
of the \texttt{oneof\{\}} fields is set. We then access the
correspondingly set field e.g. \texttt{expr.ifelse()} for the
\texttt{IfElse\_tag} case:

%{
%\href{https://github.com/mukul-rathi/bolt/blob/master/src/llvm-backend/deserialise_ir/expr_ir.cc}{expr\_ir.cc}}
%
%Copy

\begin{lstlisting}[caption={expr\_ir.cc},language=C++]
std::unique_ptr<ExprIR> deserialiseExpr(const Frontend_ir::expr &expr){
  switch (expr.tag()) {
    case Frontend_ir::expr__tag_IfElse_tag:
      return std::unique_ptr<ExprIR>(new ExprIfElseIR(expr.ifelse()));
    case Frontend_ir::expr__tag_WhileLoop_tag:
      return std::unique_ptr<ExprIR>(new ExprWhileLoopIR(expr.whileloop()));
    ...
  }
}
\end{lstlisting}

Looking at the message schema for the \texttt{\_IfElse} and
\texttt{block\_expr} messages:

%{
%\href{https://github.com/mukul-rathi/bolt/blob/master/src/frontend_ir.proto}{frontend\_ir.proto}}
%
%Copy

\begin{lstlisting}[caption={frontend\_ir.proto}]
message _IfElse {
    required expr _0 = 1;
    required block_expr _1 = 2;
    required block_expr _2 = 3;
  }
  message block_expr {
  repeated expr _ = 1;
}
\end{lstlisting}

Our constructor reads each of the \texttt{\_IfElse} message's fields in
turn. We can then use our \texttt{deserialiseExpr} to directly
deserialise the \texttt{\_0} field. However, because the message
\texttt{block\_expr} has a repeated field \texttt{\_}, we have to
iterate through the list of expr messages in that field in a for loop:

%{
%\href{https://github.com/mukul-rathi/bolt/blob/master/src/llvm-backend/deserialise_ir/expr_ir.cc}{expr\_ir.cc}}
%
%Copy

\begin{lstlisting}[language=C++,caption={expr\_ir.cc}]
ExprIfElseIR::ExprIfElseIR(const Frontend_ir::expr::_IfElse &expr) {
  Frontend_ir::expr condMsg = expr._0();
  Frontend_ir::block_expr thenBlockMsg = expr._1();
  Frontend_ir::block_expr elseBlockMsg = expr._2();

  condExpr = deserialiseExpr(condMsg);
  for (int i = 0; i < thenBlockMsg.__size(); i++) {
    thenExpr.push_back(deserialiseExpr(thenBlockMsg._(i)));
  }
  for (int i = 0; i < elseBlockMsg.__size(); i++) {
    elseExpr.push_back(deserialiseExpr(elseBlockMsg._(i)));
  }
}
\end{lstlisting}

The rest of the deserialisation code for the Bolt schema follows the
same pattern.

\hypertarget{wrap-up}{%
\section{\texorpdfstring{\protect\hyperlink{wrap-up}{}Wrap
up}{Wrap up}}\label{wrap-up}}

In this post we've converted from our OCaml frontend IR to the
equivalent C++ classes. We'll use these C++ classes to generate LLVM IR
code in the next part of the tutorial.

%\hypertarget{share-this-on-twitter}{%
%\section{Share This On Twitter}\label{share-this-on-twitter}}
%
%If you liked this post, please consider sharing it with your network. If
%you have any questions, tweet away and I'll answer :) I also tweet when
%new posts drop!
%
%\textbf{PS:} I also share helpful tips and links as I'm learning - so
%you get them \textbf{well before} they make their way into a post!
%
%\hypertarget{series-creating-the-bolt-compiler-1}{%
%\section{Series: Creating the Bolt
%Compiler}\label{series-creating-the-bolt-compiler-1}}
%
%\begin{itemize}
%\item
%  { Part 1:
%  }\href{https://mukulrathi.com/create-your-own-programming-language/intro-to-compiler/}{How
%  I wrote my own "proper" programming language}
%\item
%  { Part 2:
%  }\href{https://mukulrathi.com/create-your-own-programming-language/compiler-engineering-structure/}{So
%  how do you structure a compiler project?}
%\item
%  { Part 3:
%  }\href{https://mukulrathi.com/create-your-own-programming-language/parsing-ocamllex-menhir/}{Writing
%  a Lexer and Parser using OCamllex and Menhir}
%\item
%  { Part 4:
%  }\href{https://mukulrathi.com/create-your-own-programming-language/intro-to-type-checking/}{An
%  accessible introduction to type theory and implementing a
%  type-checker}
%\item
%  { Part 5:
%  }\href{https://mukulrathi.com/create-your-own-programming-language/data-race-dataflow-analysis/}{A
%  tutorial on liveness and alias dataflow analysis}
%\item
%  { Part 6:
%  }\href{https://mukulrathi.com/create-your-own-programming-language/lower-language-constructs-to-llvm/}{Desugaring
%  - taking our high-level language and simplifying it!}
%\item
%  \textbf{Part 7: A Protobuf tutorial for OCaml and C++}
%\item
%  { Part 8:
%  }\href{https://mukulrathi.com/create-your-own-programming-language/llvm-ir-cpp-api-tutorial/}{A
%  Complete Guide to LLVM for Programming Language Creators}
%\item
%  { Part 9:
%  }\href{https://mukulrathi.com/create-your-own-programming-language/concurrency-runtime-language-tutorial/}{Implementing
%  Concurrency and our Runtime Library}
%\item
%  { Part 10:
%  }\href{https://mukulrathi.com/create-your-own-programming-language/generics-parametric-polymorphism/}{Generics
%  - adding polymorphism to Bolt}
%\item
%  { Part 11:
%  }\href{https://mukulrathi.com/create-your-own-programming-language/inheritance-method-overriding-vtable/}{Adding
%  Inheritance and Method Overriding to Our Language}
%\end{itemize}
%
%\begin{itemize}
%\item ~
%  \hypertarget{desugaring---taking-our-high-level-language-and-simplifying-it}{%
%  \section{\texorpdfstring{\href{https://mukulrathi.com/create-your-own-programming-language/lower-language-constructs-to-llvm/}{←
%  Desugaring - taking our high-level language and simplifying
%  it!}}{← Desugaring - taking our high-level language and simplifying it!}}\label{desugaring---taking-our-high-level-language-and-simplifying-it}}
%\item ~
%  \hypertarget{tips-to-convert-your-internship-to-a-full-time-offer}{%
%  \section{\texorpdfstring{\href{https://mukulrathi.com/facebook-internship-advice/}{17
%  Tips to Convert Your Internship to a Full-time Offer!
%  →}}{17 Tips to Convert Your Internship to a Full-time Offer! →}}\label{tips-to-convert-your-internship-to-a-full-time-offer}}
%\end{itemize}
%
%\hypertarget{table-of-contents}{%
%\section{Table of Contents}\label{table-of-contents}}
%
%\href{https://mukulrathi.com/create-your-own-programming-language/protobuf-ocaml-cpp-tutorial/\#top-of-page}{}
%
%\hypertarget{a-protobuf-tutorial-for-ocaml-and-c}{%
%\section{A Protobuf tutorial for OCaml and
%C++}\label{a-protobuf-tutorial-for-ocaml-and-c}}
%
%\begin{itemize}
%\item
%  \href{https://mukulrathi.com/create-your-own-programming-language/protobuf-ocaml-cpp-tutorial/\#protocol-buffers-schema}{}
%
%  \hypertarget{protocol-buffers-schema-1}{%
%  \section{Protocol Buffers
%  Schema}\label{protocol-buffers-schema-1}}
%\item
%  \href{https://mukulrathi.com/create-your-own-programming-language/protobuf-ocaml-cpp-tutorial/\#mapping-ocaml-type-definitions-to-protobuf-schema}{}
%
%  \hypertarget{mapping-ocaml-type-definitions-to-protobuf-schema-1}{%
%  \section{Mapping OCaml Type definitions to Protobuf
%  Schema}\label{mapping-ocaml-type-definitions-to-protobuf-schema-1}}
%
%  \begin{itemize}
%  \item
%    \href{https://mukulrathi.com/create-your-own-programming-language/protobuf-ocaml-cpp-tutorial/\#ppx-deriving-protobuf}{}
%
%    \hypertarget{ppx-deriving-protobuf-1}{%
%    \section{PPX Deriving
%    Protobuf}\label{ppx-deriving-protobuf-1}}
%  \item
%    \href{https://mukulrathi.com/create-your-own-programming-language/protobuf-ocaml-cpp-tutorial/\#auto-generated-protobuf-schema}{}
%
%    \hypertarget{auto-generated-protobuf-schema-1}{%
%    \section{Auto-generated Protobuf
%    Schema}\label{auto-generated-protobuf-schema-1}}
%  \end{itemize}
%\item
%  \href{https://mukulrathi.com/create-your-own-programming-language/protobuf-ocaml-cpp-tutorial/\#decoding-protobuf-in-c}{}
%
%  \hypertarget{decoding-protobuf-in-c-1}{%
%  \section{Decoding Protobuf in
%  C++}\label{decoding-protobuf-in-c-1}}
%
%  \begin{itemize}
%  \item
%    \href{https://mukulrathi.com/create-your-own-programming-language/protobuf-ocaml-cpp-tutorial/\#generating-protobuf-deserialisation-files}{}
%
%    \hypertarget{generating-protobuf-deserialisation-files-1}{%
%    \section{Generating Protobuf Deserialisation
%    Files}\label{generating-protobuf-deserialisation-files-1}}
%  \item
%    \href{https://mukulrathi.com/create-your-own-programming-language/protobuf-ocaml-cpp-tutorial/\#deserialising-a-protobuf-serialised-file}{}
%
%    \hypertarget{deserialising-a-protobuf-serialised-file-1}{%
%    \section{Deserialising a Protobuf serialised
%    file}\label{deserialising-a-protobuf-serialised-file-1}}
%  \item
%    \href{https://mukulrathi.com/create-your-own-programming-language/protobuf-ocaml-cpp-tutorial/\#reading-out-protobuf-data-from-a-protobuf-class}{}
%
%    \hypertarget{reading-out-protobuf-data-from-a-protobuf-class-1}{%
%    \section{Reading out Protobuf data from a protobuf
%    class}\label{reading-out-protobuf-data-from-a-protobuf-class-1}}
%  \item
%    \href{https://mukulrathi.com/create-your-own-programming-language/protobuf-ocaml-cpp-tutorial/\#sanitising-our-frontend-ir}{}
%
%    \hypertarget{sanitising-our-frontend-ir-1}{%
%    \section{Sanitising our Frontend
%    IR}\label{sanitising-our-frontend-ir-1}}
%  \end{itemize}
%\item
%  \href{https://mukulrathi.com/create-your-own-programming-language/protobuf-ocaml-cpp-tutorial/\#wrap-up}{}
%
%  \hypertarget{wrap-up-1}{%
%  \section{Wrap up}\label{wrap-up-1}}
%\end{itemize}
%
%© Mukul Rathi 2023
%
%\hypertarget{gatsby-announcer}{}
%Navigated to A Protobuf tutorial for OCaml and C++
